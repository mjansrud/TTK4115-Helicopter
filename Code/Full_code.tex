
% This LaTeX was auto-generated from MATLAB code.
% To make changes, update the MATLAB code and republish this document.

\documentclass{article}
\usepackage{graphicx}
\usepackage{color}

\sloppy
\definecolor{lightgray}{gray}{0.5}
\setlength{\parindent}{0pt}

\begin{document}

    
    
\subsection*{Contents}

\begin{itemize}
\setlength{\itemsep}{-1ex}
   \item PART I
   \item PART II
   \item PART III
   \item PART IV
\end{itemize}
\begin{verbatim}
%%%%%%%%%%% Calibration of the encoder and the hardware for the specific
%%%%%%%%%%% helicopter
Joystick_gain_x = 1;
Joystick_gain_y = -1;
\end{verbatim}


\subsection*{PART I}

\begin{verbatim}
%%%%%%%%%%% Physical constants
g = 9.81; % gravitational constant [m/s^2]
l_c = 0.46; % distance elevation axis to counterweight [m]
l_h = 0.66; % distance elevation axis to helicopter head [m]
l_p = 0.175; % distance pitch axis to motor [m]
m_c = 1.92; % Counterweight mass [kg]
m_p = 0.72; % Motor mass [kg]

V_s = 7;
K_f = -(g*(m_c*l_c-2*m_p*l_h))/(V_s*l_h);
\end{verbatim}


\subsection*{PART II}

\begin{par}
Problem 1
\end{par} \vspace{1em}
\begin{verbatim}
w_0 = 2.5;

L_1 = K_f*l_p;
L_2 = g*(m_c*l_c-2*m_p*l_h);
L_3 = K_f*l_h;
L_4 = -L_3;

J_p = 2*m_p*(l_p)^2;
J_e = m_c*(l_c)^2 + 2*m_p*(l_h)^2;
J_y = m_c*(l_c)^2 + 2*m_p*((l_h)^2+(l_p)^2);

K_1 = L_1/J_p;
K_2 = L_3/J_e;
K_3 = L_4*V_s/J_y;

%%%%%%%%%%% Regulator paramaters - Method 1 - w0 og critical damping
Kpp = (w_0)^2/K_1;
Kpd = 2*sqrt(K_1*Kpp)/K_1;

% Problem 2

Krp = -1;
\end{verbatim}


\subsection*{PART III}

\begin{par}
Problem 2
\end{par} \vspace{1em}
\begin{verbatim}
A = [0 1 0; 0 0 0; 0 0 0];
B = [0 0; 0 K_1; K_2 0];
Q = [91.2 0 0; 0 50 0; 0 0 100];
R = [1 0; 0 1];
C = [1 0 0; 0 0 1];
K = lqr(A,B,Q,R);
P = inv(C*inv(-A+B*K)*B);

% Problem 3

A =  [0 1 0 0 0; 0 0 0 0 0; 0 0 0 0 0; 1 0 0 0 0; 0 0 1 0 0];
B =  [0 0; 0 K_1; K_2 0; 0 0; 0 0];
Q =  [91.2 0 0 0 0; 0 50 0 0 0; 0 0 100 0 0; 0 0 0 50 0; 0 0 0 0 50];
R =  [1 0; 0 1];
C =  [1 0 0 0 0; 0 0 1 0 0];
K =  lqr(A,B,Q,R);
P =  [0 9.999999999999998;9.549869109050656 0];
\end{verbatim}


\subsection*{PART IV}

\begin{par}
Problem 2
\end{par} \vspace{1em}
\begin{verbatim}
%%%%%%%%%%%%%% Using regulator design from part III problem 2
A_o =  [0 1 0 0 0 0; 0 0 0 0 0 0; 0 0 0 1 0 0; 0 0 0 0 0 0; 0 0 0 0 0 1; K_3 0 0 0 0 0];
B_o =  [0 0; 0 K_1; 0 0; K_2 0; 0 0; 0 0];
C_o =  [1 0 0 0 0 0; 0 0 1 0 0 0; 0 0 0 0 1 0];

%Found by testing
%poles = [-80,-80,-70+38i,-70-38i,-60+50i,-60-50i];

%Pole spreading
es = eig(A-B*K);
r0 = max(abs(es));

fr = 20;
phi = pi/10;
r = r0*fr;

spread = -phi:(phi/(2.5)):phi;
poles = -r*exp(1i*spread);

plot(real(es),imag(es),'or',real(poles),imag(poles),'kx');grid on;axis equal
L_o = place(transpose(A_o), transpose(C_o),poles).';

% Problem 3
%%%%%%%%%%%%%% Using regulator design from part III problem 2
poles = [-8,-8,-5,-20,-1,-11];
L_o = place(transpose(A_o), transpose(C_o),poles).';
\end{verbatim}

\includegraphics [width=4in]{Full_code_01.eps}



\end{document}
    
